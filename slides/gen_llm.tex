\documentclass{beamer}
\usetheme{Madrid}
\usepackage[utf8]{inputenc}
\usepackage{graphicx}
\usepackage{hyperref}

\title{Using Generative Language Models for Policy Research}
\author[E. Salas Gironés]{Edgar Salas Gironés \\ e.girones@tudelft.nl}
\date{\today}

\begin{document}

\begin{frame}
  \titlepage
\end{frame}

\begin{frame}{What Are Generative Language Models?}
\begin{itemize}
    \item AI models that generate human-like text.
    \item Examples: ChatGPT, Claude, Mistral, Gemini
    \item It is based on token-level predictions:
    \item Technically, LLMs do not understand nor reason, they only provide information based on probabilities!
\end{itemize}
\end{frame}

\begin{frame}{Alert! LLMs will always produce something.}
    LLMs will always produce something. Thus, we have two important tasks to make LLMs work `better':
    \begin{itemize}
        \item Reduce the amount of inaccurate information being produced (e.g. IPPA is taking place in Siam (rather than Thailand)).
        \item Reduce the model tendency to `create' information, i.e. hallucination (e.g. IPPA stands for International Pigeon Professionals Association)).
    \end{itemize}
\end{frame}

\begin{frame}{Why this matters?}
    \begin{itemize}
        \item You can reduce hallucinations by techniques, such as temperature, information retrieval, etc.
        \item You can reduce wrong information by update information, fact checking, human evaluation. 
    \end{itemize}
\end{frame}

\begin{frame}{Prompt Engineering 101}
    How you interact with an LLM? Through prompting. 

    \vspace{0.5cm}
    What constitutes a prompt?
    \begin{itemize}
        \item Context.
        \item Instruction.
        \item Input data (almost always)
        \item Indication of how you want the output.
    \end{itemize}
\end{frame}

% Slide 4
\begin{frame}{Chain-of-Thought Prompting}
\begin{itemize}
    \item Break complex tasks into steps
    \item Guide the model through reasoning
    \item Example:
    \begin{enumerate}
        \item Extract the main claim
        \item Identify supporting evidence
        \item Write a neutral summary
    \end{enumerate}
\end{itemize}
\end{frame}

\begin{frame}{How to do it in practice?}
    You would start small, with a basic prompt, e.g. giving only basic information and an instruction, and later refine it in small iterations.

    You would compare the results you get with a `ground truth', or human made example.
\end{frame}

\end{document}
